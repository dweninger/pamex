\section{Introduction}
\newcounter{numitem}

\label{sec:Introduction}
\vspace{\baselineskip}

\subsection{Background}
\par 
\vspace{\baselineskip}
\hspace{1em}
The objective of PAMEx is to work in conjunction with and enhance the 
pre-existing project SimpleFlow, an information-flow-based access 
control system. Even in the modern day, most access control systems 
simply disallow read or write access to confidential files. The 
SimpleFlow project was developed to go beyond most access control 
systems and instead allow malicious users to access confidential 
files up until the point of exfiltration so that SimpleFlow can monitor the user's activity. The way that 
SimpleFlow achieves this is by working on top of the Linux Security 
Module (LSM) interface and rather than simply disallowing read or write 
access like most access control systems do, SimpleFlow instead marks a 
file as either confidential or non-confidential. If a confidential 
file is wrongly accessed, the malicious user’s actions are recorded as they pass 
through the system as information flows. The wrongly accessed 
information is then prevented from being exfiltrated by SimpleFlow's network filter. 
SimpleFlow begins tracking the attacker’s actions as soon as they 
attempt to access the confidential file and therefore can capture the attacker’s 
intentions. This leaves no room for the malicious user to come up with 
a fabricated excuse as to why they were trying to exfiltrate sensitive 
information. As an example of SimpleFlow’s effectiveness, when put to 
practical use during the 2016 Cyber-Defense Exercise, the SimpleFlow 
program proved to perform well with a small overhead. While the 
practical exercise did not make use of all SimpleFlow’s functionality, 
it was able to successfully use SimpleFlow’s network filter which made 
it easy to observe any attempt at exfiltrating data. Within this 
exercise and other testing, SimpleFlow impressed attackers who were 
surprised to find their exfiltration attempts had failed \cite{ryan2016}. 

Figure~\ref{simpleflowexample} was pulled from the paper ``Studying Naive Users and the Insider Threat with SimpleFlow.''
to depict SimpleFlow being used to prevent an exfiltration attempt of sensitive data.
The example starts with a file on the kernel marked as \texttt{secret} which contains sensitive information.

\clearpage

\begin{figure}[h]
    \centering
    \includegraphics[width=.8\textwidth]{section01/assets/simpleflow_example.png}
    \caption[SimpleFlow Example]{SimpleFlow Being Used to Prevent Exfiltration of Sensitive Data (from \cite{ryan2016})\label{simpleflowexample}}
\end{figure}
\blfootnote{The above numerical SimpleFlow example items were summarized from \cite{ryan2016}.}
\begin{enumerate}[label={\protect\circled{\arabic*}}]
    \item A malicious user uses the \texttt{cat} process to open and read the confidential data in \texttt{secret}.
          Now any further actions processed by this user are tainted by SimpleFlow by being marked with an evil bit 
          and are monitored by the SimpleFlow system.
    \item The malicious user attempts to pipe and exfiltrate the information in \texttt{secret}.
    \item The pipe is forked as a child from the \texttt{cat} process and therefore is also marked as malicious activity by SimpleFlow.
    \item The \texttt{exfil} command reads \texttt{secret} and becomes tainted with the evil bit.
    \item The malicious user attepts to exfiltrate the information from \texttt{secret} to a personal server. 
          The sent packet containing \texttt{secret}'s
          information is marked with an evil bit to be easily recognizable by SimpleFlow as an unsolicited action.
    \item SimpleFlow's network filter notices the malicious user's attempt to exfiltrate the information from \texttt{secret}.
          SimpleFlow sends a fake response to the request to make the malicious user think that the process is being continued.
    \item The malicious user's exfiltration attept on \texttt{secret}'s data is ultimately blocked by SimpleFlow's network filter.
    \item The non-threatening process \texttt{wget} is performed on the system.
    \item SimpleFlow's network filter behaves normally and allows the \texttt{wget} process to continue.
\end{enumerate}


While effective, the research prototype of SimpleFlow is limited in 
that it cannot further specify the security policy of a file beyond 
being confidential and non-confidential \cite{ryan2016}. A resolution 
for SimpleFlow’s issue of a binary security system was therefore sought out and as a solution 
a customized Linux security tool suite was devised. The system would allow the 
privileged user to define hierarchical levels and non-hierarchical 
labels on a system and assign them to both system users and system 
files. This security policy was inspired by the US security 
classification system \cite{natsecinfo} by following the structure of having both levels 
and labels for security. The policy states that both system files and system users may each have one 
security level and zero or more security labels. 

As a running example, pretend that
there is a software company that is working on innovative technology that they want
to be kept secretive so that all files related to the technology are on a need-to-know basis. 
With this proposed security policy, the company could define a system saying which staff have
access to what files. For example, the privileged administrative user who oversees creating the security policy 
could define the hierarchy levels for the company from lowest to highest as
\texttt{public}, \texttt{general\_staff}, \texttt{developer}, \texttt{administrator}, and \texttt{executive\_staff}.
In this example, the company has also split the technology into several projects, 
three of which are named \texttt{alpha}, \texttt{beta}, and \texttt{charlie}. If Alice is an \texttt{administrator} for
both projects \texttt{alpha} and \texttt{beta}, then, with the newly proposed security policy,
she would be given the hierarchical level \texttt{administrator} and the non-hierarchical labels
\texttt{alpha} and \texttt{beta}. 

Continuing the above example, pretend that there is a file to outline how to begin development on project \texttt{alpha} called
\texttt{alpha}\texttt{\_dev}\texttt{\_instructions.txt}. The privileged administrator could assign \texttt{alpha}\texttt{\_dev}\texttt{\_instructions.txt} 
the security level \texttt{developer} and the security label \texttt{alpha}. Alice, who has the security level of \texttt{administrator}
which is placed higher than \texttt{developer} in the level hierarchy 
and who contains the security label \texttt{alpha} can access \texttt{alpha}\texttt{\_dev}\texttt{\_instructions.txt}.
Pretend that there is also a developer in the company named Bob who is a developer for projects \texttt{beta} and \texttt{charlie}. Bob would
have the security level \texttt{developer} which is a proper label to be able to access \texttt{alpha}\texttt{\_dev}\texttt{\_instructions.txt}, but Bob would not
be given the label \texttt{alpha} so therefore would ultimately not have access to \texttt{alpha}\texttt{\_dev}\texttt{\_instructions.txt}.

The newly proposed security policy of levels and labels 
would allow multiple groupings of users such as, in our example, 
more \texttt{developer}s who have access to project \texttt{alpha}, more \texttt{administrator}s that 
have access to multiple projects,
and files that are available to all staff members. With the newly proposed security policy, 
a system administrator would therefore be able 
to flexibly compartmentalize all the files and data that users would have 
access to on a system. Initially, it may seem like other previously-developed access control models may be sufficient
to create a security policy that is ideal for SimpleFlow such as role-based access controls (RBAC).
However, these models are not as ideal as the newly proposed PAMEx secuirty policy.
As stated above, most RBAC simply disallow reading or writing to confidential files on a system, but SimpleFlow 
wants to instead allow that access so as to be able to monitor an attacker's actions. SimpleFlow's primary job is
to record a malicious user's activity on the system and ultimately disallow exfiltrating confidential information
via its network filter. With SimpleFlow having such a particular goal, only a custom security policy such as PAMEx will allow SimpleFlow
to continue to monitor a system the way that it was designed to without interfering with SimpleFlow's actions.
Ideally, the tool suite would be developed as a standalone 
Linux Security Module so as to be tailored to but not mutually 
exclusive for SimpleFlow though. The development of the custom kernel module that would allow the tool suite to be standalone
or ported into SimpleFlow was left to future work.


\subsection{Motivation}
\par 
\vspace{\baselineskip}
\hspace{1em}
In 2022, the author reached out to his advisor, Dr. W. Michael 
Petullo showing interest in the SimpleFlow project and learning more about 
its shortcomings including the binary security policy currently in 
place. Having pursued an emphasis in cybersecurity alongside his 
degree, the author had already had an introduction to Linux security 
but wanted to learn more. The motivation of PAMEx was to aid in and 
further enhance the development of the SimpleFlow project as well as 
to learn more about Linux security at a system level and how it can 
be added to and manipulated. Building a security policy tool system 
from the ground up would give a foundational experience of 
that aspect of security systems. 

Together with his advisor, the author devised and proposed an 
alternative solution for the SimpleFlow security policy than the 
one currently in place. This policy would continue to work on top 
of the LSM interface as SimpleFlow already does and therefore be 
integrated into SimpleFlow easily. The policy’s syntax and 
tools would be user friendly and allow for much greater malleability 
while having a small footprint. Another requirement was to ensure 
that PAMEx's integration would not need to be exclusive 
to SimpleFlow but could be used on many Linux based systems. Developing 
a project of this scale is also a perfect opportunity to research, learn about, and practice the development cycle of 
a project with industry standards.  

