\section{Testing}
\label{sec:Testing}

\par 
\vspace{\baselineskip}
\hspace{1em}
Test cases were created alongside the development of PAMEx to follow 
the Agile development model and Scrum framework. For each sprint, an 
iterative completed product was to be made and therefore, that 
iterative product was tested during each sprint and specifically how 
the new iteration worked with the project as a whole \cite{lacey2015}. Unit tests were 
performed to ensure that for each function in each 
iteration, PAMEx performed how it was supposed to. These unit tests were performed manually by testing several input values for each 
parameter and by making sure to practice branch coverage by testing every 
conditional outcome. Boundary testing was also practiced during unit 
testing and therefore, edge cases were specifically paid attention to. It was 
ensured that PAMEx handled errors in the way that it was supposed to 
and even exited gracefully when need be \cite{jorgensen2021}.  

One of the major decisions made while error handling was deciding for 
each input scenario that would result in an error, should the program 
exit gracefully or ignore the input and continue. In many cases when an 
error occurred in the program, the state was unsalvageable and 
therefore would be required to exit gracefully. This was especially 
true with the PAMEx compiler. However, in some cases, such as when the 
FLUDB reads in an assignment for an entity that does not exist on the 
system such as a missing file, the program would simply print out a 
statement saying that the action could not be performed, ignore the 
action, and move on. 

Integration testing was also performed on each PAMEx program which 
focused more on the end-to-end process of the code rather than the 
individual parts \cite{jorgensen2021}. For example, the PAMEx compiler focused on 
integration testing by creating 12 end-to-end test files. Each test 
file focuses on testing a particular aspect of the PAMEx language and 
each file builds from the last. The idea is that as a new aspect of 
the PAMEx language is introduced in a new test file, if the test fails, 
the developer will know which aspect of the language is causing the 
issue. Therefore, it would be easy to pinpoint how and where 
the error occurred. Finally, a C script was developed to test 
all the compiler test files in succession.  

After performing unit tests in a similar manner to the PAMEx compiler, 
other tools had integration testing performed manually. The FLUDB tool 
was tested by creating several different policy-out files from the 
PAMEx compiler and inputting them into the tool. Once
most bugs were found using unit and integration testing and the FLUDB tool 
was in a production-level state, the policy-out files were modified to a state that the FLUDB tool was not designed to interpret to 
test its error handling.  

The PAMEx PAM module was trickier to test than the other tools because 
it is not a standalone C program. The PAM module is instead invoked 
whenever a user logs in to a system. As stated in Section 3.5 
however, if the PAM module fails during the login process, it is 
likely that the entire authentication process fails whenever a user 
attempts to authenticate on the system in the future. The lack of being able to 
authenticate on a system is a major problem. Not only can no user 
log into the system anymore, but users also cannot authenticate when 
trying to perform a process as root to fix the authentication failure. Several tools were used to 
combat this issue. First, the PAM C 
libraries \texttt{security/pam\_appl.h} and \texttt{security/pam\_misc.h} were used to create a standalone C program that performed similar 
actions to a PAM module. The C program acted like a PAM 
simulation with built in functions like \texttt{pam\_start} to begin the PAM 
process, \texttt{pam\_authenticate} to simulate a user being authenticated, and 
\texttt{pam\_end} to end the process \cite{man7pam}. Making use of these PAM libraries gave insight as to how the real PAM module should be written. With 
this insight, and by perfecting the simulation, fewer bugs were likely to be created when developing the real PAM module.  

Whenever testing or demonstrating the use of the PAM module, a virtual machine was used. The virtual machine has built-in 
functionality to snapshot a state of the machine and then reload the 
state later if necessary. Making use of this feature, a snapshot of the virtual machine was created every time a change was made 
to the PAM module but before the PAM module was fully implemented onto 
the virtual machine. This ensured that, should the PAM module break the 
virtual machine, there would be a safe state to load back to.  

During the testing of the PAM module, 
Fedora’s pamtester package was also utilized. The pamtester process allows a user to 
specify a particular grouping of PAM modules that they wish to test 
with the following command.  

\begin{verbatim}
pamtester <service-name> <username> <module-name> [<flags>] 
\end{verbatim}

In the case of the PAMEx PAM module, the pamtester command might look 
like the following line. 

\begin{verbatim}
pamtester system-auth Alice authenticate
\end{verbatim}

The test command above reads: invoke the pamtester on the \texttt{system-auth} 
file in the directory \texttt{/etc/pam.d} which contains a list of PAM modules 
to invoke in succession. Invoke the modules with the user Alice, but 
only run the PAM modules with the authentication label. The pamtester 
would then simulate all the authentication PAM modules mentioned in the 
\texttt{system- auth} file running in succession with the system user Alice being 
authenticated. Using the pamtester in this way 
directly tested the PAMEx PAM module without performing a real 
authentication process on the system and therefore, should an error 
occur, the authentication system would not break \cite{man7pam}. The pamtester was 
used extensively for integration testing. Once satisfied 
that the PAM module was working as expected, it was then tested by 
performing actual user authentication on the system including the \texttt{su} 
command, using the \texttt{sudo} command to authenticate as root, and 
graphically logging in to the system. The PAM module requires two 
parameters which specify the location of the user database file and 
the output of the PAM module and were therefore tested with several 
correct and incorrect values as well as no values at all to ensure that 
errors were handled appropriately. 

Oracle was created as a user prompt tool and therefore extensive 
integration and boundary testing was done to ensure that all user input 
was handled properly. For instance, the first value that Oracle prompts 
for is the pseudo-proc ID. If a user inputs any ID that is not found in 
the system, Oracle says that the ID cannot be found and asks the user 
to try again. A similar process is done when Oracle prompts the user 
for a command and when it prompts the user for a file. For each prompt, 
several integration and boundary tests were manually performed
until Oracle was working as expected and was in a 
production-level state. 

