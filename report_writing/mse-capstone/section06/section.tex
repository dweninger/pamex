\section{Future Work}
\label{sec:FutureWork}
\par 
\vspace{\baselineskip}
\hspace{1em}
PAMEx performs its designed job of being a custom-built 
security policy compiler which allows for more dynamic security 
policies than the default, binary policy that SimpleFlow currently 
enforces. PAMEx gives the privileged user the opportunity to create 
full hierarchical security policies on a system complete with both 
levels and labels. Although PAMEx is presented from end-to-end as a 
working security tool suite, the lack of a custom-built kernel prevents 
it from behaving as a fully-fledged security module. Therefore, 
the relationship between PAMEx and SimpleFlow as it stands is as follows:
SimpleFlow is a customized kernel and network filter, tailored to monitor the actions of malicious
users with a binary security policy of marking files as either confidential
or non-confidential. PAMEx is a linux security tool suite that allows a privileged
user to define a more flexible security policy. However, PAMEx does not have a
kernel that is customized for it and therefore cannot be automatically enforced on
a system as it stands. For PAMEx to work with SimpleFlow, SimpleFlow’s custom kernel will
need to be tailored to allow PAMEx security policies to be enforced.
In doing so, a privileged administrative user will be able to further define through PAMEx how users on a SimpleFlow system
will be marked as performing illicit activity and allow SimpleFlow to monitor their actions
and filter their exfiltration attempts. 
Using the Oracle tool to simulate the security 
policies created by PAMEx is an excellent way to show PAMEx working 
from end to end but does not allow PAMEx to be used in a fully-fledged 
production setting like SimpleFlow for which it was designed.

The file labeler tool was developed to provide the ability to quickly 
change, set, or remove the PAMEx security information from a file. In 
future work, a tool could be developed to perform similar functionality 
with user security information. With the current iteration of PAMEx, if 
a privileged, administrative user wants to add, remove, or in some way 
modify a user’s PAMEx security information, the administrator would 
need to start from the compilation process and work through each step 
of the PAMEx system. The development of a new user labeler tool would 
allow the administrator to skip some of the steps of the process should 
they find the need to modify the security information of system users. 

The need for PAMEx to be incorporated into and working 
alongside SimpleFlow still stands. Porting PAMEx into SimpleFlow is a 
significant undertaking and one that would need to be done only after a 
custom kernel has been tailored for PAMEx.
