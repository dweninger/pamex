\subsubsection*{Agile}

A project management philosophy which uses an iterative approach. Each iteration is a working product whose next iteration semantics can be changed \cite{kung2014}.

\subsubsection*{Bison}

A parser that translates a BNF grammar from the tokens that a lexical analyzer makes into groupings of statements. The statements determine which functions to perform \cite{levine2009}.

\subsubsection*{Extended Attributes}

Key-value paired metadata attached to files. Extended attributes are written to in PAMEx to store the security level and labels attached to the file. 

\subsubsection*{Flex}

The program used in PAMEx to create a lexical analyzer or scanner. Flex creates a character stream from defined, recognizable tokens and therefore outlines legal words that can be used in a system \cite{levine2009}.

\subsubsection*{Label}

A non-hierarchical compartment defined by the privileged administrative user. A file can have zero or more labels in addition to a level to further define its security. For a user to access a file with PAMEx security labels, the user must have the same corresponding labels. 

\subsubsection*{Level}

A hierarchical compartment defined by the privileged administrative user. Each level in a PAMEx system has its own placement and each file on a system can be assigned a level. To access a file with a PAMEx level, a user must be assigned a level whose placement is equivalent to the file’s level or higher.

\subsubsection*{Linux Security Module}

A framewok integrated into the Linux kernel which provides access controls to processes, files, and other aspects of the Linux system \cite{kerneldocs}.

\subsubsection*{PAMEx}

The Linux security tool suite developed to enhance the SimpleFlow project. 

\subsubsection*{Level Placement}

The hierarchical value of a level in PAMEx. Placements start at level zero for unrestricted access to a file and relationally go up by one. No two levels on the same system can share a placement value. 

\subsubsection*{Pluggable Authentication Module (PAM)}

A separated process or task which is invoked during the authentication process. PAM modules are stacked to be invoked consecutively \cite{lauber}.

\subsubsection*{Scrum}

A framework for the Agile method which uses a cyclical approach called a sprint. The sprint team consists of a scrum master, one or more product owners, and developers. Scrum outlines actions that occur to achieve an iterative product at the end of each sprint \cite{scrumorg}.

\subsubsection*{Sprint}

One cycle in the Scrum process. The PAMEx sprint time was two weeks.

\subsubsection*{sudo-proc}
The fake process file directory that PAMEx's PAM module creates. PAMEx creates this directory because it does not have privileges to write to actual process files on the system.

