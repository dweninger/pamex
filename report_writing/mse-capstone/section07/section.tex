\section{Conclusion}
\label{sec:Conclusion}
\par 
\vspace{\baselineskip}
\hspace{1em}
The PAMEx security compiler and tool suite designed to enhance the 
SimpleFlow security policy from being a binary system to being a 
flexible security system has achieved an end-to-end proof of concept. 
Being inspired by the US security classification system \cite{natsecinfo}, PAMEx can 
define both hierarchical levels and non-hierarchical labels while 
having a small footprint on a system of less than 1,700 lines 
of code. The system was developed and integrated as several lightweight 
tools. Therefore, the integration of PAMEx onto SimpleFlow or any 
other system should be straightforward because if any of the PAMEx 
tools fail, it should be simple to pinpoint the issue. 
Likewise, any future maintenance that needs to be performed on PAMEx 
can be isolated to the tool that needs to be updated or maintained and 
therefore, should also be simple. Several end-to-end demonstrations
with the aid of PAMEx's Oracle tool have proven that PAMEx works as a 
flexible, easily definable Linux security tool suite.

The development of PAMEx proved to be an invaluable experience for not 
only the further enhancement of SimpleFlow but also to 
learn from. During the development of PAMEx, a great deal was learned about the Linux security system and specifically how Linux security 
modules are invoked and function. A lot was also learned about the development of software products in general including how 
to effectively use the Agile process and Scrum framework. PAMEx uses the end-to-end simulation of the Oracle tool to prove the 
value it has for SimpleFlow and future projects. 